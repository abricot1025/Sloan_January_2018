\section{Introduction} \label{introduction}

The purpose of this project is the transfer of knowledge and technology from the MOONS trajectory computation and collision avoidance software, made by Dr. Laleh Makarem \cite{makarem_a, makarem_b, makarem_c} \& Dominique Tao, to the Sloan project, which role will be similar: compute trajectories of hundreds of actuators on a plane, with an hexagonal arrangement, to place fibers under targets, avoiding collision and deadlocks. Therefore, the two main goals are the update of the program for the Sloan case, and the improvement of the convergence rate (that is, the number of actuator able to reach their assigned target without any collision or deadlocks).
\\

During this project, I mainly worked on the first goal. This report aim at explaining the work I have done during this semester, as well as the work still ongoing. To do so, a reminder explaining the differences between the two projects will be given. Then, Sections \ref{circles} and \ref{parameters} will explain the change in the architecture of the software, especially the introduction of a program generating the actuators structures for the Sloan project as well as the centralisation of every parameters in a single text file. Section \ref{set_target} shows the update of the target setting function and the fiber attribution, regarding to the major changes on the Sloan project, being the adding of new fibers on each ferules. In Section \ref{target_generator}, a small script enabling the generation of target files to test the software will be presented. Section \ref{visualisation} shows the improvements implemented on the animation available at the end of the computation. Finally, Sections \ref{to-do} and \ref{conclusion} will describe ongoing jobs as well as the rest of the tasks to be done to obtain a working software for the Sloan project. Changes in the code of the software can be seen in the Appendices.